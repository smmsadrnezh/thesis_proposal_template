
%--------------------------------------------------------------------------
%
%  LaTeX Thesis Proposal Template (v200919)
%  By: S. M. Masoud Sadrnezhaad
%
%--------------------------------------------------------------------------

\documentclass[letterpaper,11pt]{article}

%--------------------------------------------------------------------------
%
%  LaTeX Thesis Proposal Template (v200919)
%  By: S. M. Masoud Sadrnezhaad
%
%--------------------------------------------------------------------------

\usepackage{fullpage,amsmath,hyperref,color,clrscode,amsfonts,graphicx}
\usepackage{listings}
\usepackage{hyperref}
\usepackage{float}

% Checked Squares %
\usepackage{pifont}
\usepackage{amssymb}
\newcommand{\checkedsquare}{\rlap{\raisebox{0.3ex}{\hspace{0.4ex}\tiny \ding{52}}}$\square$}

% Frame Around Text %
\usepackage{mdframed}
\newmdenv[skipbelow=0pt]{abovestacked}
\newmdenv[skipabove=0pt]{belowstacked}

% Nested Enumerate Numbering %
\usepackage{enumitem}

\addtolength{\textwidth}{0.2in}
\addtolength{\oddsidemargin}{-0.1in}
\addtolength{\evensidemargin}{-0.1in}

\addtolength{\textheight}{0.5in}
\addtolength{\topmargin}{-0.25in}

\usepackage{adjustbox}
\usepackage{setspace}
\usepackage[localise=on]{xepersian}

% Bibliography with no title %
\usepackage[backend=biber,style=numeric]{biblatex}
\renewcommand\refname{}
\addbibresource{bibs.bib}
\DeclareBibliographyCategory{primary}
\DeclareBibliographyCategory{secondary}

% Font Settings %
\settextfont[Scale=1]{XB Yas}[
  Path=./assets/,
  BoldFont=* Bd,
  ItalicFont=* It,
  BoldItalicFont=* BdIt,
  Extension=.ttf
]
\setlatintextfont[Scale=0.8]{Arial}
\setsayehfont{XB Kayhan Sayeh]}[
  Path=./assets/,
  Extension=.ttf
]
\begin{document}
\begin{center}
\footnotesize{بسمه تعالی} \\ \vspace{20pt}
\includegraphics[scale=0.4]{assets/logo.png} \\
\footnotesize{\textbf{دانشکدهٔ مهندسی و علوم کامپیوتر}}\\ \vspace{10pt}
{\LARGE \textbf{طرح پیشنهادی پایان‌نامه کارشناسی ارشد}} \\
\end{center}
\begin{enumerate}
\item \textbf{مشخصات دانشجو}
\begin{table}[H]
\begin{center}
\begin{tabular*}{0.94\textwidth}{|r@{\extracolsep{\fill}} r r |}
\hline
\textbf{نام و نام خانوادگی:} لورم ایپسوم & \textbf{شماره دانشجویی:} ۹۸۱۲۳۴۵۶ & \textbf{سال ورود:} ۱۳۹۸ \ \\
\hline
\textbf{رشتهٔ تحصیلی:} فناوری اطلاعات & \textbf{گرایش:} معماری سازمانی & \textbf{دوره:} \checkedsquare روزانه $\square$ شبانه $\square$ پردیس \\
\hline
\end{tabular*}
\end{center}
\end{table}
\item \textbf{مشخصات استادان راهنما و مشاور}
\begin{table}[H]
\begin{center}
\begin{tabular*}{0.94\textwidth}{|r@{\extracolsep{\fill}} r |}
\hline
 \textbf{نام و نام‌خانوادگی استادراهنمای‌اول:} لورم ایپسوم & \textbf{تخصص اصلی:} لورم ایپسوم \\
\hline
\textbf{آخرین مدرک تحصیلی:} دکتری & \textbf{رتبه دانشگاهی:} استادیار \\
\hline
\end{tabular*}
\end{center}
\end{table}
\item \textbf{اطلاعات کلی پایان‌نامه / رساله}
\begin{enumerate}
  \item \textbf{عنوان}
\begin{table}[H]
\begin{center}
\begin{tabular*}{0.94\textwidth}{|c | @{\extracolsep{\fill}} c |}
\hline
 \textbf{فارسی} & لورم ایپسوم متن ساختگی \\
\hline
\textbf{انگلیسی} & \lr{Lorem ipsum dolor sit amet} \\
\hline
\end{tabular*}
\end{center}
\end{table}
  \item \textbf{واژگان کلیدی (بین سه تا پنج مورد)}
\begin{table}[H]
\begin{center}
\begin{tabular*}{0.94\textwidth}{|c | @{\extracolsep{\fill}} c |}
\hline
 \textbf{فارسی} & لورم، ایپسوم، متن ساختگی \\
\hline
\textbf{انگلیسی} & \lr{Lorem, ipsum, dolor, sit amet} \\
\hline
\textbf{نوع تحقیق} & $\square$ بنیادی \checkedsquare نظری \checkedsquare کاربردی $\square$ توسعه‌ای \\
\hline
\end{tabular*}
\end{center}
\end{table}
\end{enumerate}
\clearpage
\item \textbf{جزییات پیشنهاد}
\begin{enumerate}
  \item \textbf{تعریف مسئله و سؤال‌های اصلی تحقیق}
\begin{mdframed}
\textbf{تعریف مسئله} \newline
لورم ایپسوم متن ساختگی با تولید سادگی نامفهوم از صنعت چاپ و با استفاده از طراحان گرافیک است. چاپگرها و متون بلکه روزنامه و مجله در ستون و سطرآنچنان که لازم است و برای شرایط فعلی تکنولوژی مورد نیاز و کاربردهای متنوع با هدف بهبود ابزارهای کاربردی می‌باشد. کتابهای زیادی در شصت و سه درصد گذشته، حال و آینده شناخت فراوان جامعه و متخصصان را می‌طلبد تا با نرم‌افزارها شناخت بیشتری را برای طراحان رایانه ای علی‌الخصوص طراحان خلاقی و فرهنگ پیشرو در زبان فارسی ایجاد کرد. در این صورت می‌توان امید داشت که تمام و دشواری موجود در ارائه راهکارها و شرایط سخت تایپ به پایان رسد وزمان مورد نیاز شامل حروفچینی دستاوردهای اصلی و جوابگوی سوالات پیوسته اهل دنیای موجود طراحی اساساً مورد استفاده قرار گیرد. \newline
\textbf{سوال‌های اصلی} \newline
لورم ایپسوم متن ساختگی با تولید سادگی نامفهوم از صنعت چاپ و با استفاده از طراحان گرافیک است. چاپگرها و متون بلکه روزنامه و مجله در ستون و سطرآنچنان که لازم است و برای شرایط فعلی تکنولوژی مورد نیاز و کاربردهای متنوع با هدف بهبود ابزارهای کاربردی می‌باشد. کتابهای زیادی در شصت و سه درصد گذشته، حال و آینده شناخت فراوان جامعه و متخصصان را می‌طلبد تا با نرم‌افزارها شناخت بیشتری را برای طراحان رایانه ای علی‌الخصوص طراحان خلاقی و فرهنگ پیشرو در زبان فارسی ایجاد کرد. در این صورت می‌توان امید داشت که تمام و دشواری موجود در ارائه راهکارها و شرایط سخت تایپ به پایان رسد وزمان مورد نیاز شامل حروفچینی دستاوردهای اصلی و جوابگوی سوالات پیوسته اهل دنیای موجود طراحی اساساً مورد استفاده قرار گیرد.
\end{mdframed}
  \item \textbf{پیشینه پژوهش}
\begin{mdframed}
لورم ایپسوم متن ساختگی با تولید سادگی نامفهوم از صنعت چاپ و با استفاده از طراحان گرافیک است. چاپگرها و متون بلکه روزنامه و مجله در ستون و سطرآنچنان که لازم است و برای شرایط فعلی تکنولوژی مورد نیاز و کاربردهای متنوع با هدف بهبود ابزارهای کاربردی می‌باشد. کتابهای زیادی در شصت و سه درصد گذشته، حال و آینده شناخت فراوان جامعه و متخصصان را می‌طلبد تا با نرم‌افزارها شناخت بیشتری را برای طراحان رایانه ای علی‌الخصوص طراحان خلاقی و فرهنگ پیشرو در زبان فارسی ایجاد کرد. در این صورت می‌توان امید داشت که تمام و دشواری موجود در ارائه راهکارها و شرایط سخت تایپ به پایان رسد وزمان مورد نیاز شامل حروفچینی دستاوردهای اصلی و جوابگوی سوالات پیوسته اهل دنیای موجود طراحی اساساً مورد استفاده قرار گیرد.
\end{mdframed}
  \item \textbf{رویکرد حل مسئله}
\begin{mdframed}
لورم ایپسوم متن ساختگی با تولید سادگی نامفهوم از صنعت چاپ و با استفاده از طراحان گرافیک است. چاپگرها و متون بلکه روزنامه و مجله در ستون و سطرآنچنان که لازم است و برای شرایط فعلی تکنولوژی مورد نیاز و کاربردهای متنوع با هدف بهبود ابزارهای کاربردی می‌باشد. کتابهای زیادی در شصت و سه درصد گذشته، حال و آینده شناخت فراوان جامعه و متخصصان را می‌طلبد تا با نرم‌افزارها شناخت بیشتری را برای طراحان رایانه ای علی‌الخصوص طراحان خلاقی و فرهنگ پیشرو در زبان فارسی ایجاد کرد. در این صورت می‌توان امید داشت که تمام و دشواری موجود در ارائه راهکارها و شرایط سخت تایپ به پایان رسد وزمان مورد نیاز شامل حروفچینی دستاوردهای اصلی و جوابگوی سوالات پیوسته اهل دنیای موجود طراحی اساساً مورد استفاده قرار گیرد.
\end{mdframed}
  \item \textbf{فرضیه‌ها (هر فرضیه به صورت جمله خبری نوشته شود)}
\begin{mdframed}
لورم ایپسوم متن ساختگی با تولید سادگی نامفهوم از صنعت چاپ و با استفاده از طراحان گرافیک است. چاپگرها و متون بلکه روزنامه و مجله در ستون و سطرآنچنان که لازم است و برای شرایط فعلی تکنولوژی مورد نیاز و کاربردهای متنوع با هدف بهبود ابزارهای کاربردی می‌باشد. کتابهای زیادی در شصت و سه درصد گذشته، حال و آینده شناخت فراوان جامعه و متخصصان را می‌طلبد تا با نرم‌افزارها شناخت بیشتری را برای طراحان رایانه ای علی‌الخصوص طراحان خلاقی و فرهنگ پیشرو در زبان فارسی ایجاد کرد. در این صورت می‌توان امید داشت که تمام و دشواری موجود در ارائه راهکارها و شرایط سخت تایپ به پایان رسد وزمان مورد نیاز شامل حروفچینی دستاوردهای اصلی و جوابگوی سوالات پیوسته اهل دنیای موجود طراحی اساساً مورد استفاده قرار گیرد.
\end{mdframed}
  \item \textbf{جنبه جدید بودن و نوآوری طرح در چیست؟} \newline
این مورد باید با کمک استاد محترم راهنما کامل شود و به تأیید ایشان برسد.
\begin{mdframed}
لورم ایپسوم متن ساختگی با تولید سادگی نامفهوم از صنعت چاپ و با استفاده از طراحان گرافیک است. چاپگرها و متون بلکه روزنامه و مجله در ستون و سطرآنچنان که لازم است و برای شرایط فعلی تکنولوژی مورد نیاز و کاربردهای متنوع با هدف بهبود ابزارهای کاربردی می‌باشد. کتابهای زیادی در شصت و سه درصد گذشته، حال و آینده شناخت فراوان جامعه و متخصصان را می‌طلبد تا با نرم‌افزارها شناخت بیشتری را برای طراحان رایانه ای علی‌الخصوص طراحان خلاقی و فرهنگ پیشرو در زبان فارسی ایجاد کرد. در این صورت می‌توان امید داشت که تمام و دشواری موجود در ارائه راهکارها و شرایط سخت تایپ به پایان رسد وزمان مورد نیاز شامل حروفچینی دستاوردهای اصلی و جوابگوی سوالات پیوسته اهل دنیای موجود طراحی اساساً مورد استفاده قرار گیرد. \newline\newline
\textbf{امضا استاد راهنما}
\end{mdframed}
  \item \textbf{روش و ابزارهای لازم برای ارزیابی ایدهٔ ارائه شده (تحلیل، شبیه‌سازی، آمارگیری و …)}
\begin{mdframed}
\textbf{روش ارزیابی} \newline
لورم ایپسوم متن ساختگی با تولید سادگی نامفهوم از صنعت چاپ و با استفاده از طراحان گرافیک است. چاپگرها و متون بلکه روزنامه و مجله در ستون و سطرآنچنان که لازم است و برای شرایط فعلی تکنولوژی مورد نیاز و کاربردهای متنوع با هدف بهبود ابزارهای کاربردی می‌باشد. کتابهای زیادی در شصت و سه درصد گذشته، حال و آینده شناخت فراوان جامعه و متخصصان را می‌طلبد تا با نرم‌افزارها شناخت بیشتری را برای طراحان رایانه ای علی‌الخصوص طراحان خلاقی و فرهنگ پیشرو در زبان فارسی ایجاد کرد. در این صورت می‌توان امید داشت که تمام و دشواری موجود در ارائه راهکارها و شرایط سخت تایپ به پایان رسد وزمان مورد نیاز شامل حروفچینی دستاوردهای اصلی و جوابگوی سوالات پیوسته اهل دنیای موجود طراحی اساساً مورد استفاده قرار گیرد. \newline
\textbf{مقالات مشابه} \newline
لورم ایپسوم متن ساختگی با تولید سادگی نامفهوم از صنعت چاپ و با استفاده از طراحان گرافیک است. چاپگرها و متون بلکه روزنامه و مجله در ستون و سطرآنچنان که لازم است و برای شرایط فعلی تکنولوژی مورد نیاز و کاربردهای متنوع با هدف بهبود ابزارهای کاربردی می‌باشد. کتابهای زیادی در شصت و سه درصد گذشته، حال و آینده شناخت فراوان جامعه و متخصصان را می‌طلبد تا با نرم‌افزارها شناخت بیشتری را برای طراحان رایانه ای علی‌الخصوص طراحان خلاقی و فرهنگ پیشرو در زبان فارسی ایجاد کرد. در این صورت می‌توان امید داشت که تمام و دشواری موجود در ارائه راهکارها و شرایط سخت تایپ به پایان رسد وزمان مورد نیاز شامل حروفچینی دستاوردهای اصلی و جوابگوی سوالات پیوسته اهل دنیای موجود طراحی اساساً مورد استفاده قرار گیرد. \newline
\textbf{ابزار پیاده‌سازی} \newline
\begin{itemize}
  \item \lr{Python}
\end{itemize}
\end{mdframed}
  \item \textbf{لیست مراجع حمایت کننده موضوع پیشنهادی} \newline
  با مراجعه به تارنمای دانشکده (منوی ارتباط با صنعت زیر منوی حمایت از پایان‌نامه‌ها و رساله‌ها)، لیست ارگان‌هایی که موضوع مورد پیشنهادی را حمایت می‌کنند، استخراج نمایید. با توجه به اینکه مراحل بعدی حمایت از پایان‌نامه یا رساله پس از تصویب این پیشنهاد طبق همین لیست انجام خواهد شد، لازم است این لیست با دقت و پس از بررسی موضوعات مورد حمایت ارگان‌ها تکمیل گردد و به امضای استاد راهنما برسد.
\begin{mdframed}
لورم ایپسوم متن ساختگی با تولید سادگی نامفهوم از صنعت چاپ و با استفاده از طراحان گرافیک است. چاپگرها و متون بلکه روزنامه و مجله در ستون و سطرآنچنان که لازم است و برای شرایط فعلی تکنولوژی مورد نیاز و کاربردهای متنوع با هدف بهبود ابزارهای کاربردی می‌باشد. کتابهای زیادی در شصت و سه درصد گذشته، حال و آینده شناخت فراوان جامعه و متخصصان را می‌طلبد تا با نرم‌افزارها شناخت بیشتری را برای طراحان رایانه ای علی‌الخصوص طراحان خلاقی و فرهنگ پیشرو در زبان فارسی ایجاد کرد. در این صورت می‌توان امید داشت که تمام و دشواری موجود در ارائه راهکارها و شرایط سخت تایپ به پایان رسد وزمان مورد نیاز شامل حروفچینی دستاوردهای اصلی و جوابگوی سوالات پیوسته اهل دنیای موجود طراحی اساساً مورد استفاده قرار گیرد. \newline\newline
\textbf{امضا استاد راهنما}
\end{mdframed}
\end{enumerate}
\item \textbf{فهرست منابع و مآخذ (فارسی، عربی، انگلیسی با قالب زیر)}
\begin{mdframed}
\renewcommand{\thempfootnote}{\arabic{mpfootnote}}
\begin{latin}
\baselineskip=.8\baselineskip
\nocite{*}
\begin{center}\end{center}\vspace{-2em}
\printbibliography[heading=none]
\end{latin}
\end{mdframed}
\item \textbf{جدول زمانبندی مراحل انجام پایان‌نامه (از زمان تصویب تا دفاع نهایی)} \newline
موارد و زمانبندی ذکر شده در جدول زیر به عنوان نمونه ارائه شده‌اند و لازم است متناسب با موضوع پیشنهادی و با کمک استاد محترم راهنما اصلاح شوند.
\begin{table}[H]
\begin{center}
\begin{tabular*}{0.95\textwidth}{@{\extracolsep{\fill}} | c | p{5.41cm} | c | c | c | c | c | c | c | c | c | c | c | c |}
\hline
ردیف & شرح فعالیت‌ها & ۱ & ۲ & ۳ & ۴ & ۵ & ۶ & ۷ & ۸ & ۹ & ۱۰ & ۱۱ & ۱۲ \\
\hline
۱ & جستجوی مقالات مرتبط و مطالعه آن‌ها & \checkedsquare & & & & & & & & & & & \\
\hline
۲ & استخراج مفاهیم موردنیاز و چالش‌های مطرح & & \checkedsquare & & & & & & & & & & \\
\hline
۳ & بررسی مدل‌ها و راه حل‌های مطرح & & \checkedsquare & \checkedsquare & & & & & & & & & \\
\hline
۴ & پیشنهاد مدل & & & & \checkedsquare & \checkedsquare & \checkedsquare & & & & & & \\
\hline
۵ & تأیید و اعتبارسنجی مدل & & & & & & & \checkedsquare & \checkedsquare & & & & \\
\hline
۶ & مطالعهٔ موردی & & & & & & & & \checkedsquare & & & & \\
\hline
۷ & نگارش مقاله & & & & & & & & & \checkedsquare & \checkedsquare & & \\
\hline
۸ & نگارش پایان‌نامه & & & & & & & & & & & \checkedsquare & \checkedsquare \\
\hline
\end{tabular*}
\end{center}
\end{table}
\clearpage
\item \textbf{محل امضاء}
\begin{table}[H]
\begin{center}
\begin{tabular*}{0.94\textwidth}{@{\extracolsep{\fill}} | r | r | r |}
\hline
\textbf{نام و نام خانوادگی دانشجو:} & \textbf{امضا} & \textbf{تاریخ} \\
  & & \\
\hline
\textbf{نام و نام خانوادگی استاد راهنما:} & \textbf{امضا} & \textbf{تاریخ} \\
  & & \\
\hline
\textbf{نام و نام خانوادگی استاد راهنمای دوم و مشاور:} & \textbf{امضا} & \textbf{تاریخ} \\
  & & \\
\hline
\end{tabular*}
\end{center}
\end{table}
* دقت نمایید که علاوه بر بند ۷، بندهای ۴–۵ و ۴–۷ نیز باید به تأیید استاد محترم راهنما برسد.
\item \textbf{صورت‌جلسهٔ شورای گروه آموزشی}

موضوع تحقیق پایان‌نامه/رساله خانم/آقای ........................................................................... \newline
دانشجوی مقطع کارشناسی ارشد \checkedsquare دکتری $\square$ رشته ............................................................... \newline
تحت عنوان: .............................................................................................................. \newline
در جلسه مورخ: ...... / ...... / ......... شورای گروه .............................................. مطرح شد و \newline
$\square$ به همین صورت مورد تصویب قرار گرفت. \newline
$\square$ با تغییراتی به شرح زیر مورد تصویب قرار گرفت: \newline
............................................................................................................................... \newline
............................................................................................................................... \newline
............................................................................................................................... \newline
............................................................................................................................... \newline
............................................................................................................................... \newline
............................................................................................................................... \newline
............................................................................................................................... \newline
............................................................................................................................... \newline
............................................................................................................................... \newline
............................................................................................................................... \newline
$\square$ مورد تصویب قرار نگرفت. \newline
محل امضای داوران محترم:
\begin{table}[H]
\begin{center}
\begin{tabular*}{0.94\textwidth}{| c | @{\extracolsep{\fill}} c | c |}
\hline
\textbf{نام و نام خانوادگی داور} & \textbf{امضا} & \textbf{تاریخ} \\
\hline
 & & \\
 & & \\
\hline
 & & \\
 & & \\
\hline
 & & \\
 & & \\
\hline
 & & \\
 & & \\
\hline
\textbf{نام و نام خانوادگی مدیرگروه} & \textbf{امضا} & \textbf{تاریخ} \\
 & & \\
 & & \\
\hline
\end{tabular*}
\end{center}
\end{table}

\end{enumerate}
\end{document}